\documentclass[lualatex,js=standard,hiresbb,openany]{bxjsarticle}
\usepackage{luatexja}
\usepackage{geometry}
\geometry{margin=17mm}

%数式
\usepackage{amsmath,amsfonts}
\usepackage{bm}
\usepackage{siunitx}
\usepackage{ulem}
\usepackage{mathtools}

%画像
\usepackage[dvipdfmx]{graphicx}
\usepackage{color}
\usepackage{subfigure}
\usepackage{float}
\usepackage{caption}

%枠付き文章
\usepackage{ascmac}
\usepackage{comment}
\usepackage{url}


%松尾先生用
\begin{comment}
\usepackage{fancyhdr}
\pagestyle{fancy}
   \lhead{情報工学実験 第4章 デコーダおよび数字表示回路に関する実験}
   \chead{\empty} %ヘッダ中央
   \rhead{\empty} %ヘッダ右.コンパイルした日付を表示
   \lfoot{\empty} %フッタ左
   \cfoot{\thepage} %フッタ中央.ページ番号を表示
   \rfoot{\empty} %フッタ右
   \renewcommand{\headrulewidth}{0pt}

\end{comment}
\begin{document}


\center{掛け算プログラム(3 $\times$ 10と 4 $\times$ 3 を計算して,両結果を減算)}

\begin{table}[H]
  \begin{center}
   \begin{tabular}{lcccc}\hline
      アドレス & ニーモニック & マシンコード & コメント\\
      \hline \hline
      0 & MOV GR2, 0    & 0011 10 00 000000 & レジスタ2に0を代入\\
      1 & MOV GR0, 10   & 0011 00 00 001010 & レジスタ0に10を代入\\
      2 & SUB GR0, 0    & 0111 00 00 000000 & レジスタ0から0を減算\\
      3 & JZE at 7      & 1010 00 00 000010 & 前の計算結果でZero Flag = 1なら\\ &&&当該アドレスにジャンプ\\
      4 & ADD GR2, 3    & 0101 10 00 000011 & レジスタ2に3を加算\\
      5 & SUB GR0, 1    & 0111 00 00 000001 & レジスタ0から1を減算\\
      6 & JMP at 2      & 0001 00 00 000010 & 当該アドレスに強制ジャンプ\\
      7 & MOV GR3, 0    & 0011 11 00 000000 & レジスタ3に0を代入\\
      8 & MOV GR0, 3    & 0011 00 00 000011 & レジスタ0に3を代入\\
      9 & SUB GR0, 0    & 0111 00 00 000000 & レジスタ0から0を減算\\
     10 & JZE at 14     & 1010 00 00 001110 & 前の計算結果でZero Flag = 1なら\\ &&&当該アドレスにジャンプ\\
     11 & ADD GR3, 4    & 0101 11 00 000100 & レジスタ3に4を加算\\
     12 & SUB GR0, 1    & 0111 00 00 000001 & レジスタ0から1を減算\\
     13 & JMP at 9      & 0001 00 00 001001 & 当該アドレスに強制ジャンプ\\
     14 & SUB GR2, GR3  & 0110 10 11 000000 & GR2とGR3を減算 \\
     15 & MOV GR1, GR2  & 0010 01 10 000000 & GR1にGR2の値をコピー\\
     16 & OUT GR1       & 1111 01 00 000000 & 出力レジスタにGR1の値を出力\\
     17 & MOV GR0, 0    & 0011 00 00 000000 & レジスタ0に0を代入\\
     18 & MOV GR1, 0    & 0011 01 00 000000 & レジスタ1に0を代入\\
     19 & MOV GR2, 0    & 0011 10 00 000000 & レジスタ2に0を代入\\
     20 & MOV GR3, 0    & 0011 11 00 000000 & レジスタ3に0を代入\\
     21 & OUT GR0       & 1111 00 00 000000 & 出力レジスタにGR0の値を出力\\
     22 & JMP at 0      & 0001 00 00 000000 & 当該アドレスに強制ジャンプ\\
   

     
     \hline                          
   \end{tabular}
   \label{tab:}
  \end{center}
 \end{table}
 \newpage

 \center{符号付6bit 2進数対応表}
 \begin{table}[H]
  \begin{center}
   \begin{tabular}{|c|c|c|c|c|c|c|c|}\hline
      10進数 & 2進数 & & & 10進数 & 2進数 & & \\
      \hline
      0      & 000000 &  16 & 010000 0  &   -1      & 111111 &  -17 & 101111 \\
      1      & 000001 &  17 & 010001    &   -2      & 111110 &  -18 & 101110 \\     
      2      & 000010 &  18 & 010010    &   -3      & 111101 &  -19 & 101101 \\   
      3      & 000011 &  19 & 010011    &   -4      & 111100 &  -20 & 101100 \\   
      4      & 000100 &  20 & 010100    &   -5      & 111011 &  -21 & 101011 \\   
      5      & 000101 &  21 & 010101    &   -6      & 111010 &  -22 & 101010 \\   
      6      & 000110 &  22 & 010110    &   -7      & 111001 &  -23 & 101001 \\   
      7      & 000111 &  23 & 010111    &   -8      & 111000 &  -24 & 101000 \\   
      8      & 001000 &  24 & 011000    &   -9      & 110111 &  -25 & 100111 \\   
      9      & 001001 &  25 & 011001    &  -10      & 110110 &  -26 & 100110 \\   
     10      & 001010 &  26 & 011010    &  -11      & 110101 &  -27 & 100101 \\   
     11      & 001011 &  27 & 011011    &  -12      & 110100 &  -28 & 100100 \\   
     12      & 001100 &  28 & 011100    &  -13      & 110011 &  -29 & 100011 \\   
     13      & 001101 &  29 & 011101    &  -14      & 110010 &  -30 & 100010 \\   
     14      & 001110 &  30 & 011110    &  -15      & 110001 &  -31 & 100001 \\   
     15      & 001111 &  31 & 011111    &  -16      & 110000 &  -32 & 100000 \\   
     \hline
   \end{tabular}
   \label{tab:}
  \end{center}
 \end{table}

 

\end{document}



